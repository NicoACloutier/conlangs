\documentclass{article}[10pt]

\usepackage[a4paper, total={6in, 10in}]{geometry}
\usepackage{tabularx}
\usepackage{textcomp}
\usepackage{amsmath}
\usepackage{graphicx}
\usepackage{amssymb}
\usepackage{tipa}
\usepackage{multicol}
\usepackage{gb4e}
\usepackage{titling}
\usepackage{tabularray}
\usepackage{qtree}
\usepackage{amssymb}
\usepackage{vowel}

\newcommand{\subtitle}[1]{%
  \posttitle{%
    \par\end{center}
    \begin{center}\large#1\end{center}
    \vskip0.3em}%
}

\newcommand{\subauthor}[1]{%
  \postauthor{%
    \par\end{center}
    \begin{center}\large#1\end{center}
    \vskip0.3em}%
}

\newcommand{\define}[4]{\emph{#1} [ \textsc{#2} $\rightarrow$ \textsc{#3} ] : #4. \\}
\newcommand{\defarg}[2]{\emph{#1}: ``#2.''\\}

\title{Neritary Grammar\\with Texts and Vocabulary}
\subtitle{
\emph{Cane riNeri ciTary}
%\vspace{0.3cm} \\ \includegraphics[scale=0.22]{title2.png}
}
\author{Nicolas Antonio Cloutier}

\begin{document}
\maketitle

\vspace{0.25in}

%{\begin{center}
%\includegraphics[scale=0.25]{circleflag.png} \end{center}}

\vspace{0.25in}

{\begin{center}
\emph{}\\
\vspace{0.5cm}
--
\end{center}}

\clearpage
{\bf \emph{1 Phonology}}\\

\begin{center}
\emph{Table I: Consonants}
\begin{tabular}{ |c|c|c|c|c|c|c| }
\hline
 & \bf{Labial} & \bf{Alveolar} & \bf{Postalveolar} & \bf{Palatal} & \bf{Velar} & \bf{Glottal} \\ \hline
\bf{Plosive} & p & t d & & & k g & \textipa{P} \\ \hline
\bf{Nasal} & m  &  n  & & \textltailn  &  & \\ \hline
\bf{Trill} & & r & & & & \\\hline
\bf{Fricative} & f & s & \textesh & &  & h \\ \hline
\bf{Approximant} & w & l & & & (w) &  \\ \hline
\end{tabular}
\end{center}

\begin{center}
\emph{Table II: Romanization of consonants}
\begin{tabular}{ |c|c|c|c|c|c|c| }
\hline
 & \bf{Labial} & \bf{Alveolar} & \bf{Postalveolar} & \bf{Palatal} & \bf{Velar} & \bf{Glottal} \\ \hline
\bf{Plosive} & p & t d & & & c/qu g/gu & ' \\ \hline
\bf{Nasal} & m & n & \~{n} & & & \\ \hline
\bf{Trill} & & r & & & & \\\hline
\bf{Fricative} & f & z/c & x & & & j \\ \hline
\bf{Approximant} & hu & l & & & & \\ \hline
\end{tabular}
\end{center}

\begin{multicols}{2}
{\begin{center}
\emph{Figure I: Vowels}\\
 \Large
\begin{vowel}
	\putcvowel[l]{i}{1}
	\putcvowel[l]{e}{2}
	\putcvowel[l]{a}{4}
	\putcvowel[l]{\textipa{1}}{9}
\end{vowel} \end{center}}

{\begin{center}
\emph{Figure II: Romanization of vowels}\\
 \Large
\begin{vowel}
	\putcvowel[l]{i}{1}
	\putcvowel[l]{e}{2}
	\putcvowel[l]{a}{4}
	\putcvowel[l]{y}{9}
\end{vowel} \end{center}}
\end{multicols}

In unstressed syllables, /a/ is prounced as [\textipa{v}], and /i/ as [\textipa{I}]. Spanish rules are followed when multiple romanizations given. For example, /si/ is written as $\langle$ci$\rangle$, but /sa/ is written as $\langle$za$\rangle$, /gi/ is written as $\langle$gui$\rangle$ but /ga/ is written as $\langle$ga$\rangle$. All syllables are (C)V. Adjacent vowels are treated as nuclei of separate syllables. Stress can be varied, and is marked by the acute diacritic, unless stress is on the ultimate syllable of a multisyllabic word. If a monosyllabic word receives stress, its vowel is marked with an acute. At the end of each sentence (marked in writing with a period, exclamation point, or question mark), a glottal stop is added to the coda of the final word (which is the one exception to the otherwise entirely (C)V phonology). This goes unwritten.

\clearpage
{\bf \emph{2 Grammar}}\\

{\bf 2.1 Class marking}

To mark an argument as belonging to a class, its name is added to the beginning of the argument phrase.

{\bf 2.2 The existance statement \emph{o}}

The particle \emph{o} marks an existance statement, which claims the existance of a particular thing. It is the simplest type of Eritary phrase, and can be constructed with only a single argument.

\begin{exe}
\ex
\gll o $\varnothing$-je\\
\textsc{exist} \textsc{human}-human\\
\trans \emph{A human exists.}
\end{exe}

{\bf 2.3 The intransitive predicate-NP statement \emph{a}}

The particle \emph{a} marks an intransitive predicate-NP statement, which denotes that a particular NP is the sole argument of an intransitive predicate. Notably, this does not cover a case where a multi-argument transitive predicate is used, but other arguments are implied, meaning only one argument is used; it is only to be used when only one argument is intended in the statement. The first argument begins immediately after the particle is used and is expected to be of the \textsc{Action} class. No action class prefix need be provided. The second comes after an intermediary particular \emph{'ae} and is in the \textsc{human} class by default, but can be marked for a different class.\\ 

\begin{exe}
\ex
\gll a $\varnothing$-tary 'ae $\varnothing$-je\\
\textsc{int.pred-np} \textsc{action}-speak \textsc{2\textsuperscript{nd}.arg} \textsc{human}-human\\
\trans \emph{A human speaks.}
\end{exe}

\clearpage
{\bf \emph{3 Semantics and Lexicon}}\\

There are five semantic classes: the \textsc{Human} class, only for humans, the \textsc{Action} class, for actions that can be carried out, the \textsc{Animate} non-human class, for animals, the \textsc{Concept} class, for abstract concepts, and the \textsc{Inanimate} class, for non-abstract, non-animate physical objects. Arguments can be broken into these classes, with cognate arguments in different semantic classes having different but often related meanings. Similarly, a single predicate can have several different but related meanings when taking differnent numbers of inputs and from different classes. These are defined in section 3.2, along with their class signatures. There is also a sixth semantic class: the \textsc{Foreign} class, for loan words, numbers, and proper nouns.\\

{\bf 3.1 Arguments}\\

\emph{3.1.1 The human class}
\begin{multicols}{3}
\noindent
\defarg{je}{The generic argument; a human}
\defarg{teme}{woman}
\defarg{\~{n}awa}{man}
\end{multicols}

\emph{3.1.2 The action class: \emph{na}}
\begin{multicols}{3}
\noindent
\defarg{cate}{To compute}
\defarg{je}{The generic argument; to do}
\defarg{xeca}{To track the time}
\defarg{teme}{To give birth}
\defarg{tary}{To speak}
\end{multicols}

\emph{3.1.3 The animate non-human class: \emph{ta}}
\begin{multicols}{3}
\noindent
\defarg{je}{The generic argument; an animal}
\defarg{teme}{female}
\defarg{\~{n}awa}{male}
\end{multicols}

\emph{3.1.4 The concept class: \emph{que}}
\begin{multicols}{3}
\noindent
\defarg{cate}{Mathematics and computation}
\defarg{je}{The generic argument; a concept}
\defarg{xeca}{Time}
\defarg{tary}{Human speech}
\defarg{teme}{Femininity}
\defarg{\~{n}awa}{Masculinity}
\end{multicols}

\emph{3.1.5 The inanimate class: \emph{ri}}
\begin{multicols}{3}
\noindent
\defarg{cate}{Computer}
\defarg{je}{The generic argument; a thing}
\defarg{xeca}{Clock}
\end{multicols}

{\bf 3.2 Predicates}

\noindent
\define{cate}{Human}{Action}{To understand someone}
\define{ne}{Foreign}{Human}{A person given their name}
\define{neri}{Foreign}{Concept}{A language given its name}
\define{tary}{Human}{Concept}{Someone's speech}
\define{xeque}{Action}{Union\{Human, Animate\}}{The performer of an action}
\define{xeque}{Animal}{Animal}{The parent of an animal}
\define{xeque}{Concept}{Concept}{The origin of a concept}
\define{xeque}{Human}{Human}{The parent of a person}
\define{xeque}{Inanimate}{Union\{Human, Animate\}}{The creator of an object}

\clearpage
{\bf \emph{4 Short texts}}\\

\end{document}
