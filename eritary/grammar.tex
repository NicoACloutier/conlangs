\documentclass{article}[10pt]

\usepackage[a4paper, total={6in, 10in}]{geometry}
\usepackage{tabularx}
\usepackage{textcomp}
\usepackage{amsmath}
\usepackage{graphicx}
\usepackage{amssymb}
\usepackage{tipa}
\usepackage{multicol}
\usepackage{gb4e}
\usepackage{titling}
\usepackage{tabularray}
\usepackage{qtree}
\usepackage{amssymb}
\usepackage{vowel}

\newcommand{\subtitle}[1]{%
  \posttitle{%
    \par\end{center}
    \begin{center}\large#1\end{center}
    \vskip0.3em}%
}

\newcommand{\subauthor}[1]{%
  \postauthor{%
    \par\end{center}
    \begin{center}\large#1\end{center}
    \vskip0.3em}%
}

\newcommand{\define}[4]{\emph{#1} [ \textsc{#2} $\rightarrow$ \textsc{#3} ] : #4. \\}
\newcommand{\defarg}[2]{\emph{#1}: ``#2.''\\}

\title{Eritary Grammar\\with Texts and Vocabulary}
\subtitle{
\emph{Ca'e rEri ciTary}
%\vspace{0.3cm} \\ \includegraphics[scale=0.22]{title2.png}
}
\author{Nicolas Antonio Cloutier}

\begin{document}
\maketitle

\vspace{0.25in}

%{\begin{center}
%\includegraphics[scale=0.25]{circleflag.png} \end{center}}

\vspace{0.25in}

{\begin{center}
\emph{}\\
\vspace{0.5cm}
--
\end{center}}

\clearpage
{\bf \emph{1 Phonology}}\\

\begin{center}
\emph{Table I: Consonants}
\begin{tabular}{ |c|c|c|c|c|c|c| }
\hline
 & \bf{Labial} & \bf{Alveolar} & \bf{Postalveolar} & \bf{Palatal} & \bf{Velar} & \bf{Glottal} \\ \hline
\bf{Plosive} & p & t d & & & k g & \textipa{P} \\ \hline
\bf{Nasal} & m  &  n  & & \textltailn  &  & \\ \hline
\bf{Trill} & & r & & & & \\\hline
\bf{Fricative} & f & s & \textesh & &  & h \\ \hline
\bf{Approximant} & w & l & & & (w) &  \\ \hline
\end{tabular}
\end{center}

\begin{center}
\emph{Table II: Romanization of consonants}
\begin{tabular}{ |c|c|c|c|c|c|c| }
\hline
 & \bf{Labial} & \bf{Alveolar} & \bf{Postalveolar} & \bf{Palatal} & \bf{Velar} & \bf{Glottal} \\ \hline
\bf{Plosive} & p & t d & & & c/qu g/gu & ' \\ \hline
\bf{Nasal} & m & n & \~{n} & & & \\ \hline
\bf{Trill} & & r & & & & \\\hline
\bf{Fricative} & f & z/c & x & & & j \\ \hline
\bf{Approximant} & hu & l & & & & \\ \hline
\end{tabular}
\end{center}

\begin{multicols}{2}
{\begin{center}
\emph{Figure I: Vowels}\\
 \Large
\begin{vowel}
	\putcvowel[l]{i}{1}
	\putcvowel[l]{e}{2}
	\putcvowel[l]{a}{4}
	\putcvowel[l]{\textipa{1}}{9}
\end{vowel} \end{center}}

{\begin{center}
\emph{Figure II: Romanization of vowels}\\
 \Large
\begin{vowel}
	\putcvowel[l]{i}{1}
	\putcvowel[l]{e}{2}
	\putcvowel[l]{a}{4}
	\putcvowel[l]{y}{9}
\end{vowel} \end{center}}
\end{multicols}

In unstressed syllables, /a/ is prounced as [\textipa{v}], and /i/ as [\textipa{I}]. Spanish rules are followed when multiple romanizations given. For example, /si/ is written as $\langle$ci$\rangle$, but /sa/ is written as $\langle$za$\rangle$, /gi/ is written as $\langle$gui$\rangle$ but /ga/ is written as $\langle$ga$\rangle$. All syllables are (C)V. Adjacent vowels are treated as nuclei of separate syllables. Stress can be varied, and is marked by the acute diacritic, unless stress is on the penultimate syllable of a multisyllabic word. If a monosyllabic word receives stress, its vowel is marked with an acute.

\clearpage
{\bf \emph{2 Grammar}}\\

{\bf 2.1 Class marking}

    To mark an argument as belonging to a class, a (possibly shortened) version of its name is added to the beginning of the argument phrase. If phonosyntactically allowed, \emph{j\'{u}} becomes \emph{j-}, \emph{nace} \emph{n(a)-}, \emph{tani} \emph{t(a)-}, \emph{qu\'{e}} \emph{c-}, \emph{rina} \emph{r(i)-}, and \emph{cie} \emph{z/ci-}.  

\clearpage
{\bf \emph{3 Semantics and Lexicon}}\\

There are five semantic classes: the \textsc{Human} class, only for humans, the \textsc{Action} class, for actions that can be carried out, the \textsc{Animate} non-human class, for animals, the \textsc{Concept} class, for abstract concepts, and the \textsc{Inanimate} class, for non-abstract, non-animate physical objects. Arguments can be broken into these classes, with cognate arguments in different semantic classes having different but often related meanings. Similarly, a single predicate can have several different but related meanings when taking differnent numbers of inputs and from different classes. These are defined in section 3.2, along with their class signatures. There is also a sixth semantic class: the \textsc{Foreign} class, for loan words and proper nouns.\\

{\bf 3.1 Arguments}\\

\emph{3.1.1 The human class: \emph{j\'{u}}}
\begin{multicols}{3}
\noindent
\defarg{je}{The generic argument; a human}
\defarg{teme}{woman}
\defarg{\~{n}awa}{man}
\end{multicols}

\emph{3.1.2 The action class: \emph{nace}}
\begin{multicols}{3}
\noindent
\defarg{cate}{To compute}
\defarg{je}{The generic argument; to do}
\defarg{xeca}{To track the time}
\defarg{teme}{To give birth}
\defarg{tary}{To speak}
\end{multicols}

\emph{3.1.3 The animate non-human class: \emph{tani}}
\begin{multicols}{3}
\noindent
\defarg{je}{The generic argument; an animal}
\defarg{teme}{female}
\defarg{\~{n}awa}{male}
\end{multicols}

\emph{3.1.4 The concept class: \emph{qu\'{e}}}
\begin{multicols}{3}
\noindent
\defarg{cate}{Mathematics and computation}
\defarg{je}{The generic argument; a concept}
\defarg{xeca}{Time}
\defarg{tary}{Human speech}
\defarg{teme}{Femininity}
\defarg{\~{n}awa}{Masculinity}
\end{multicols}

\emph{3.1.5 The inanimate class: \emph{rina}}
\begin{multicols}{3}
\noindent
\defarg{cate}{Computer}
\defarg{je}{The generic argument; a thing}
\defarg{xeca}{Clock}
\end{multicols}

{\bf 3.2 Predicates}

\noindent
\define{cate}{Human}{Action}{To understand someone}
\define{ca'e}{Concept}{Inanimate}{A book about something}
\define{eri}{Foreign}{Concept}{A language given its name}
\define{'e}{Foreign}{Human}{A person given their name}
\define{tary}{Human}{Concept}{Someone's speech}
\define{xeque}{Action}{Union\{Human, Animate\}}{The performer of an action}
\define{xeque}{Animal}{Animal}{The parent of an animal}
\define{xeque}{Concept}{Concept}{The origin of a concept}
\define{xeque}{Human}{Human}{The parent of a person}
\define{xeque}{Inanimate}{Union\{Human, Animate\}}{The creator of an object}

\clearpage
{\bf \emph{4 Short texts}}\\

\end{document}
